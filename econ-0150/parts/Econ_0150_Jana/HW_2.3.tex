\documentclass[12pt]{article}
\usepackage[margin=1in]{geometry}
\usepackage{enumitem}
\usepackage{parskip}
\usepackage{fancyhdr}
\usepackage{titlesec}
\usepackage{url}
\usepackage{graphicx}

% Header/Footer
\pagestyle{fancy}
\fancyhf{}
\rhead{ECON 0150 | Spring 2025}
\lhead{Homework 2.2}
\rfoot{\thepage}

% Section formatting
\titleformat{\section}{\large\bfseries}{}{0em}{}
\titleformat{\subsection}[runin]{\bfseries}{}{0em}{}[.]

\begin{document}

Homework is designed to both test your knowledge and challenge you to apply familiar concepts in new applications. Answer clearly and completely. You are welcomed and encouraged to work in groups so long as your work is your own. Use the provided datasets to answer the following questions. Then submit your figures and answers to Gradescope.

\section{Q1 US Cities Analysis}
Using the provided dataset of US cities that includes:

\begin{itemize}
    \item City name
    \item Population
    \item Latitude/longitude coordinates
    \item Average temperature
\end{itemize}

\subsection*{a)}
Plot all city locations (using lat/lng) on a scatterplot (which represents a map in this case). In a separate color, on the same figure, plot all cities in the 'America/New\_York' timezone. *(Interesting but not required: represent the population of the city with the size of the point.)

\subsection*{b)} Create a scatter plot showing the relationship between latitude and average temperature for cities in the 'America/New\_York' timezone. How would you describe this relationship?

\begin{itemize}
    \item Strongly Positive
    \item Weakly Positive
    \item Unclear
    \item Weakly Negative
    \item Strongly Negative
\end{itemize}















\end{document}