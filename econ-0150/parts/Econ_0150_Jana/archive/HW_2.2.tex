\documentclass[12pt]{article}
\usepackage[margin=1in]{geometry}
\usepackage{enumitem}
\usepackage{parskip}
\usepackage{fancyhdr}
\usepackage{titlesec}
\usepackage{url}
\usepackage{graphicx}

% Header/Footer
\pagestyle{fancy}
\fancyhf{}
\rhead{ECON 0150 | Spring 2025}
\lhead{Homework 2.2}
\rfoot{\thepage}

% Section formatting
\titleformat{\section}{\large\bfseries}{}{0em}{}
\titleformat{\subsection}[runin]{\bfseries}{}{0em}{}[.]

\begin{document}

\vspace*{-1cm}
\noindent \textbf{ECON 0150 | Spring 2025 | Homework 2.2} \\
\textbf{Due: } \\

\noindent Homework is designed to both test your knowledge and challenge you to apply familiar concepts in new applications. \\
Answer clearly and completely. You are welcomed and encouraged to work in groups so long as your work is your own. Use the datafile to answer the following questions. Then submit your figures and answers to Gradescope.

\section*{Q1: Relationships Through Time}

Using the dataset \texttt{coffee\_prod\_in\_years.csv}, which provides information on coffee production and employment in agriculture in different countries between 1961 and 2023:

\subsection*{a)} Plot a line graph of global coffee production over time (the total across all countries for each year). Describe the trend and discuss potential reasons behind it.

\subsection*{b)} Plot a line graph of coffee production for each country over time. Include all countries in a single figure.

\subsection*{c)} Generate a scatter plot comparing coffee production in 1961 and 2023 for each country. Include a 45-degree line to help identify which countries increased or decreased production. Are there outlier countries? Briefly suggest why these countries might stand out.

\subsection*{d)} Which figure — the country-level time trends (from part b) or the 1961 vs. 2023 comparison (from part c) — better helps you understand patterns in coffee production? Justify your answer.

\subsection*{e)} Choose one of the outlier countries you identified in part (c) and plot its coffee production over time. Describe the pattern and propose possible explanations (e.g., conflict, policy changes, new entrants to the market).







\end{document}